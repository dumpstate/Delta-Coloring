\documentclass[a4paper]{article}
\usepackage{polski}
\usepackage[utf8]{inputenc}
\usepackage[margin=1in]{geometry}
\usepackage{listings}
\usepackage{graphicx}
\usepackage{amsmath}

\title{\textbf{Algorytmy Zaawansowane - kolorowanie $Delta(G)$ instrukcja użytkownika}}
\author{\textbf{Albert Sadowski, Piotr Stanek}\\
	Wydział Matematyki i Nauk Informacyjnych\\
	Politechnika Warszawska}

\begin{document}
	\maketitle

	{\em Delta coloring} jest aplikacja konsolowa realizującą algorytm kolorowania grafu o czasie wykonania ograniczonym przez {\em Delta(G)}.

	Aplikacja przyjmuje jako argument graf ścieżkę do pliku zawierającego graf zdefiniowany w formacie DOT. Plik wykonywalny {\em deltaColoring} do starndardowego wyjścia drukuje wynik realizacji algorytmu - graf w formacie DOT ze znacznikami odpowiadającymi za kolor poszczególnych wierzchołków grafu.

	Przykładowe uruchomienie programu:

	\begin{verbatim}

		deltaColoring /path/to/input.dot

	\end{verbatim}

	Skrypt {\em run.sh} uruchamia aplikację {\em deltaColoring} wraz z innymi programami odpowiadającymi za wygenerowania pliku PDF z pokolorowanym grafem wyjściowym - rezultatem programu. 

	Przykładowe uruchomienie skryptu:

	\begin{verbatim}

		run.sh /path/to/input.dot

	\end{verbatim}


	Skrypt korzysta z aplikacji {\em dot}, która rysuje graf podany w formacie DOT do formatu post script, oraz z aplikacji {\em ps2pdf}, która dla zadanego pliku w formacie post script generuje plik PDF. Skrypt automatycznie uruchamia program {\em evince} z wyjściowym grafem.

	Wyjściowy plik PDF ({\em graph.pdf}) zapisany jest w folderze {\em output} wraz z plikiem {\em err.log} odpowiadającym za zapis strumienia {\em stderr}.

\end{document}
